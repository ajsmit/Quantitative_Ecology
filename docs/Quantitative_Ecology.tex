\documentclass[english,10pt,a4paper,oneside]{book}
\usepackage[english]{babel}

% PAGE GEOMETRY
\usepackage{geometry}
\geometry{margin=4cm}

% FONTS
% with lining figures for math roman; load before other text specs:
% \usepackage{eulervm}
\usepackage{amssymb,amsmath}
\usepackage{ifxetex}
\ifxetex
  \usepackage{microtype}
  \usepackage{mathspec} % only with xelatex; before fontspec
  \usepackage{fontspec} % only with xelatex
  % \defaultfontfeatures{Ligatures=Common}
  \setmainfont[Ligatures=TeX]{Minion Pro}
  \setsansfont[Ligatures=TeX,Scale=MatchLowercase]{Myriad Pro}
  \setmonofont[Scale=MatchLowercase]{Fira Code}
  % custom ampersand
  \newcommand{\amper}{{\fontspec[Scale=.95]{Hoefler Text}\selectfont\itshape\&}}
\else
  \usepackage[activate={true,nocompatibility},final,tracking=true,kerning=true,spacing=true,factor=1100,stretch=10,shrink=10]{microtype}
  \usepackage[T1]{fontenc}
  \usepackage[utf8]{inputenc}
  \DeclareUnicodeCharacter{2212}{-}
\fi

% TYPOGRAPHY SETTINGS
% disable protrusion for tt fonts
\UseMicrotypeSet[protrusion]{basicmath}
% remove excessive space after full-stop
\frenchspacing

% % CHAPTER HEADINGS
% \usepackage{sectsty} % don't work with 'quotchap'
% \chapterfont{\usefont{T1}{qhv}{b}{n}\selectfont\huge}
% qotation ahead of chapters and fancy chapter heading
\usepackage{quotchap}
\renewcommand*{\sectfont}{\sffamily\bfseries\Huge\selectfont}

% SECTION, SUBSECTION, ETC. TITLES
\usepackage[compact]{titlesec}
% \titleformat{\chapter} % don't work with 'quotchap'
%   {\normalfont\Huge\sffamily\bfseries}
%   {\thechapter}
%   {1em}
%   {}
\titleformat{\section}
  {\normalfont\LARGE\sffamily\bfseries}
  {\thesection}
  {1em}
  {}
\titleformat{\subsection}
  {\normalfont\Large\sffamily\bfseries}
  {\thesubsection}
  {1em}
  {}
\titleformat{\subsubsection}
  {\normalfont\large\sffamily\bfseries\slshape}
  {\thesubsubsection}
  {1em}
  {}
% \titlespacing*{<command>}{<left>}{<before-sep>}{<after-sep>}
\titlespacing*{\section}
  {0pt}
  {1.2ex plus 1ex minus .2ex}
  {0.5ex plus .1ex minus .1ex}
\titlespacing*{\subsection}
  {0pt}
  {1.2ex plus 1ex minus .2ex}
  {0.5ex plus .1ex minus .1ex}
\titlespacing*{\subsubsection}
  {0pt}
  {1.2ex plus 1ex minus .2ex}
  {0.5ex plus .1ex minus .1ex}

% FIGURE AND TABLE CAPTIONS
\usepackage{floatrow}
\floatsetup[figure]{capposition=bottom}
\floatsetup[table]{capposition=top}

% QUOTATIONS AND QUOTATION MARKS
\usepackage[autostyle]{csquotes} % works with babel
% use upquote if available, for straight quotes in verbatim environments
\usepackage{upquote}{}

% HYPERLINKS
\usepackage{hyperref}
\hypersetup{
    colorlinks=true,
    linkcolor=blue,
    filecolor=magenta,
    urlcolor=cyan,
}
\urlstyle{same}
% avoid problems with \sout in headers with hyperref
\usepackage[normalem]{ulem}
\pdfstringdefDisableCommands{\renewcommand{\sout}{}}

% FOOTNOTES
\usepackage{fancyvrb}
\VerbatimFootnotes % allows verbatim text in footnotes
% Make links footnotes instead of hotlinks
\renewcommand{\href}[2]{#2\footnote{\url{#1}}}
% Use protect on footnotes to avoid problems with footnotes in titles
\let\rmarkdownfootnote\footnote%
\def\footnote{\protect\rmarkdownfootnote}

% BIBLIOGRAPHY
\usepackage{natbib}

% TABLES
\usepackage{longtable,booktabs,tabularx,ragged2e,dcolumn,multirow,multicol}
\setlength\heavyrulewidth{0.1em}
\setlength\lightrulewidth{0.0625em}

% SI UNITS
\usepackage{siunitx}
    \sisetup{%
        detect-mode,
        group-digits            = false,
        input-symbols           = ( ) [ ] - + < > * §,
        table-align-text-post   = false,
        round-mode              = places,
        round-precision         = 3
        }

% GRAPHICS SETTINGS
\usepackage{graphicx,grffile}
\makeatletter
\def\maxwidth{\ifdim\Gin@nat@width>\linewidth\linewidth\else\Gin@nat@width\fi}
\def\maxheight{\ifdim\Gin@nat@height>\textheight\textheight\else\Gin@nat@height\fi}
\makeatother
% Scale images if necessary, so that they will not overflow the page
% margins by default, and it is still possible to overwrite the defaults
% using explicit options in \includegraphics[width, height, ...]{}
\setkeys{Gin}{width=\maxwidth,height=\maxheight,keepaspectratio}

% LINESPACING
% \usepackage{setspace}
% \setstretch{1.5}

% NO INDENTATION WITH A SPACE BETWEEN PARAGRAPHS
\usepackage{parskip}
\setlength{\parindent}{0pt}
\setlength{\parskip}{6pt plus 2pt minus 1pt}
% prevent overfull lines
\setlength{\emergencystretch}{3em}
\providecommand{\tightlist}{%
  \setlength{\itemsep}{0pt}\setlength{\parskip}{0pt}}

% THE NEXT LINES OF CODE SPECIFY SOME YAML ENTRIES %
%%%%%%%%%%%%%%%%%%%%%%%%%%%%%%%%%%%%%%%%%%%%%%%%%%%%
% switch yes/no in YAML header

% ENABLES LISTING OF CODE (echo = TRUE)
% listings: yes/no
\usepackage{color}
\usepackage{fancyvrb}
\newcommand{\VerbBar}{|}
\newcommand{\VERB}{\Verb[commandchars=\\\{\}]}
\DefineVerbatimEnvironment{Highlighting}{Verbatim}{commandchars=\\\{\}}
% Add ',fontsize=\small' for more characters per line
\usepackage{framed}
\definecolor{shadecolor}{RGB}{255,255,255}
\newenvironment{Shaded}{\begin{snugshade}}{\end{snugshade}}
\newcommand{\KeywordTok}[1]{\textcolor[rgb]{0.12,0.11,0.11}{\textbf{#1}}}
\newcommand{\DataTypeTok}[1]{\textcolor[rgb]{0.00,0.34,0.68}{#1}}
\newcommand{\DecValTok}[1]{\textcolor[rgb]{0.69,0.50,0.00}{#1}}
\newcommand{\BaseNTok}[1]{\textcolor[rgb]{0.69,0.50,0.00}{#1}}
\newcommand{\FloatTok}[1]{\textcolor[rgb]{0.69,0.50,0.00}{#1}}
\newcommand{\ConstantTok}[1]{\textcolor[rgb]{0.67,0.33,0.00}{#1}}
\newcommand{\CharTok}[1]{\textcolor[rgb]{0.57,0.30,0.62}{#1}}
\newcommand{\SpecialCharTok}[1]{\textcolor[rgb]{0.24,0.68,0.91}{#1}}
\newcommand{\StringTok}[1]{\textcolor[rgb]{0.75,0.01,0.01}{#1}}
\newcommand{\VerbatimStringTok}[1]{\textcolor[rgb]{0.75,0.01,0.01}{#1}}
\newcommand{\SpecialStringTok}[1]{\textcolor[rgb]{1.00,0.33,0.00}{#1}}
\newcommand{\ImportTok}[1]{\textcolor[rgb]{1.00,0.33,0.00}{#1}}
\newcommand{\CommentTok}[1]{\textcolor[rgb]{0.54,0.53,0.53}{#1}}
\newcommand{\DocumentationTok}[1]{\textcolor[rgb]{0.38,0.47,0.50}{#1}}
\newcommand{\AnnotationTok}[1]{\textcolor[rgb]{0.79,0.38,0.79}{#1}}
\newcommand{\CommentVarTok}[1]{\textcolor[rgb]{0.00,0.58,1.00}{#1}}
\newcommand{\OtherTok}[1]{\textcolor[rgb]{0.00,0.43,0.16}{#1}}
\newcommand{\FunctionTok}[1]{\textcolor[rgb]{0.39,0.29,0.61}{#1}}
\newcommand{\VariableTok}[1]{\textcolor[rgb]{0.00,0.34,0.68}{#1}}
\newcommand{\ControlFlowTok}[1]{\textcolor[rgb]{0.12,0.11,0.11}{\textbf{#1}}}
\newcommand{\OperatorTok}[1]{\textcolor[rgb]{0.12,0.11,0.11}{#1}}
\newcommand{\BuiltInTok}[1]{\textcolor[rgb]{0.39,0.29,0.61}{\textbf{#1}}}
\newcommand{\ExtensionTok}[1]{\textcolor[rgb]{0.00,0.58,1.00}{\textbf{#1}}}
\newcommand{\PreprocessorTok}[1]{\textcolor[rgb]{0.00,0.43,0.16}{#1}}
\newcommand{\AttributeTok}[1]{\textcolor[rgb]{0.00,0.34,0.68}{#1}}
\newcommand{\RegionMarkerTok}[1]{\textcolor[rgb]{0.00,0.34,0.68}{#1}}
\newcommand{\InformationTok}[1]{\textcolor[rgb]{0.69,0.50,0.00}{#1}}
\newcommand{\WarningTok}[1]{\textcolor[rgb]{0.75,0.01,0.01}{#1}}
\newcommand{\AlertTok}[1]{\textcolor[rgb]{0.75,0.01,0.01}{\textbf{#1}}}
\newcommand{\ErrorTok}[1]{\textcolor[rgb]{0.75,0.01,0.01}{\underline{#1}}}
\newcommand{\NormalTok}[1]{\textcolor[rgb]{0.12,0.11,0.11}{#1}}

% NUMBERED SECTIONS OR NOT
% numbersections: yes/no
\setcounter{secnumdepth}{5}

% REDEFINE SUBPARAGRAPHS
% subparagraph: yes
% Redefines (sub)paragraphs to behave more like sections
\ifx\paragraph\undefined\else
\let\oldparagraph\paragraph
\renewcommand{\paragraph}[1]{\oldparagraph{#1}\mbox{}}
\fi
\ifx\subparagraph\undefined\else
\let\oldsubparagraph\subparagraph
\renewcommand{\subparagraph}[1]{\oldsubparagraph{#1}\mbox{}}
\fi

% COMPACT TITLES
% CREATE SUBTITLE COMMAND FOR USE IN MAKETITLE
\usepackage{titling}
\newcommand{\subtitle}[1]{
  \posttitle{
    \begin{center}\large#1\end{center}
    }
}

% YAML ENTRIES: 'title', 'subtitle', 'author' and 'date'
\setlength{\droptitle}{-2em}
  \title{Quantitative Ecology}
  \pretitle{\vspace{\droptitle}\centering\huge}
  \posttitle{\par}
\subtitle{\sf A primer in ecological field and computer techniques for BCB (Hons) 2018}
  \author{AJ Smit}
  \preauthor{\centering\large\emph}
  \postauthor{\par}
  \predate{\centering\large\emph}
  \postdate{\par}
  \date{2018-01-22}

% UNCOMMENT IF FIGURE ABOVE TITLE IS NOT REQUIRED
\pretitle{%
  \begin{center}
  \LARGE
  \includegraphics[width=5cm]{figures/769_life_finds_a_way.png}\\[\bigskipamount]
}
\posttitle{\end{center}}

% ANY HEADER/PREAMBLE LaTeX code can be added to the file
% specified here by
% header-includes: path/to/the/file.tex
% HEADERS AND FOOTERS
\usepackage{fancyhdr}
\pagestyle{fancy}
% \renewcommand{\footrule}{\hrule height 0.8pt \vspace{2mm}}
\fancyhead[L]{R for Data Analysis}
\fancyhead[R]{\includegraphics[width=1cm]{figures/Rlogo.png}}
\fancyfoot[L]{Rhodes}
\fancyfoot[C]{4 September -- 8 September 2017}
\fancyfoot[R]{\thepage}

% CODE FONT
\usepackage{xcolor}
\definecolor{ttcolor}{RGB}{255,110,120} % salmon
% redefine \texttt
\let\Oldtexttt\texttt
\renewcommand\texttt[1]{{\ttfamily\color{ttcolor}#1}}

% BOXES
\usepackage{tcolorbox}
\tcbuselibrary{breakable}
\newtcolorbox{mybox}[3][]
{
  fonttitle = \bfseries\itshape\small,
  fontupper = \small,
  halign    = flush left,
  colframe  = gray!15,
  colback   = gray!5!white,
  coltitle  = black!5!black,
  boxrule   = 1.5pt,
  before    = {\vspace{0.20cm}},
  title     = #3,
  arc       = 0mm,
  #1,
  breakable,
}

\usepackage{amsthm}
\newtheorem{theorem}{Theorem}[chapter]
\newtheorem{lemma}{Lemma}[chapter]
\theoremstyle{definition}
\newtheorem{definition}{Definition}[chapter]
\newtheorem{corollary}{Corollary}[chapter]
\newtheorem{proposition}{Proposition}[chapter]
\theoremstyle{definition}
\newtheorem{example}{Example}[chapter]
\theoremstyle{definition}
\newtheorem{exercise}{Exercise}[chapter]
\theoremstyle{remark}
\newtheorem*{remark}{Remark}
\newtheorem*{solution}{Solution}
\begin{document}
\maketitle



% TABLE OF CONTENT
% toc: yes/no
{
\hypersetup{linkcolor=black}
\setcounter{tocdepth}{1}
\tableofcontents
}

% LIST OF TABLES
\listoftables

% LIST OF FIGURES
\listoffigures

\chapter*{Prerequisites}\label{prerequisites}
\addcontentsline{toc}{chapter}{Prerequisites}

A prerequisite for this course is a basic proficiency in using R
\citep{R2017}. The necessary experience will have been gained from
completing the
\href{https://robwschlegel.github.io/Intro_R_Workshop/}{Intro R
Workshop: Data Manipulation, Analysis and Graphing} Workshop that was
part of your BCB Core Honours module (i.e.~Biostatistics). You will also
need a laptop with R and RStudio installed as per the instructions
provided in that workshop. If you do not have a personal laptop, most
computers in the 5th floor lab will be correctly set up for this
purpose.

\chapter*{About this document}\label{about-this-document}
\addcontentsline{toc}{chapter}{About this document}

This document, which as available as a HTML file that's viewable on a
web browser of your choice (anything will do, but we discourage using
Internet Explorer) and as a PDF (accessible from the link at the top of
any of the website's pages) that may be printed, was prepared by the
software tools available to R via RStudio. We use the package called
\texttt{bookdown} that may be accessed and read about
\href{https://bookdown.org/yihui/bookdown/}{here} to produce this
documentation. The entire source code to reproduce this book is
available from my \href{https://github.com/ajsmit/}{GitHub account}.

You will notice that this repository uses
\href{https://github.com}{GitHub}, and you are advised to set up your
own repository for R scripts and all your data. We will touch on GitHub
and the principles of reproducible research later, and GitHub forms a
core ingredient of such a workflow.

\includegraphics[width=2.78in]{figures/bookdown_hex_logo}

The R session information when compiling this book is shown below:

\begin{Shaded}
\begin{Highlighting}[]
\KeywordTok{sessionInfo}\NormalTok{()}
\end{Highlighting}
\end{Shaded}

\begin{verbatim}
## R version 3.4.3 (2017-11-30)
## Platform: x86_64-apple-darwin15.6.0 (64-bit)
## Running under: macOS Sierra 10.12.6
## 
## Matrix products: default
## BLAS: /Library/Frameworks/R.framework/Versions/3.4/Resources/lib/libRblas.0.dylib
## LAPACK: /Library/Frameworks/R.framework/Versions/3.4/Resources/lib/libRlapack.dylib
## 
## locale:
## [1] en_US.UTF-8/en_US.UTF-8/en_US.UTF-8/C/en_US.UTF-8/en_US.UTF-8
## 
## attached base packages:
## [1] stats     grDevices utils     datasets 
## [5] graphics  base     
## 
## loaded via a namespace (and not attached):
##  [1] Rcpp_0.12.15    bookdown_0.5   
##  [3] tufte_0.2       png_0.1-7      
##  [5] digest_0.6.14   rprojroot_1.3-2
##  [7] backports_1.1.2 formatR_1.5    
##  [9] magrittr_1.5    evaluate_0.10.1
## [11] stringi_1.1.6   rmarkdown_1.8  
## [13] tools_3.4.3     stringr_1.2.0  
## [15] yaml_2.1.16     compiler_3.4.3 
## [17] htmltools_0.3.6 knitr_1.18     
## [19] methods_3.4.3
\end{verbatim}

\chapter*{Preliminaries}\label{preliminaries}
\addcontentsline{toc}{chapter}{Preliminaries}

\begin{quote}
\emph{\enquote{You can know the name of that bird in all the languages
of the world, but when you're finished, you'll know absolutely nothing
whatever about the bird. You'll only know about humans in different
places, and what they call the bird. \ldots{} I learned very early the
difference between knowing the name of something and knowing
something.}}

--- Richard Feynman
\end{quote}

\section*{Venue, date and time}\label{venue-date-and-time}
\addcontentsline{toc}{section}{Venue, date and time}

Quantitative Ecology is scheduled to replace Plant Ecophysiology, and
will run between May 14th and June 29th, 2018. This workshop will take
place from \textbf{9:00--16:00} on one day in each week during this
period. There will also be a field component, where you will be taught
about ecological field sampling in marine and terrestrial environments;
this will also offer an opportunity to collect real data using actual
ecological field methods (these will also be covered in the course),
which we will then analyse using the multivariate methods used in this
workshop.

\section*{Course outline}\label{course-outline}
\addcontentsline{toc}{section}{Course outline}

\begin{itemize}
\tightlist
\item
  Week 1 -- In the Beginning
\item
  Week 2 -- Show and tell
\item
  Week 3 -- Going deeper
\item
  Week 4 -- The Enlightened Researcher
\item
  Week 5 -- The world is yours
\end{itemize}

\section*{About this Workshop}\label{about-this-workshop}
\addcontentsline{toc}{section}{About this Workshop}

The aim of this five-day introductory workshop is to guide you
through\ldots{}

\section*{This is biology: why more R
coding?}\label{this-is-biology-why-more-r-coding}
\addcontentsline{toc}{section}{This is biology: why more R coding?}

Please refer to the
\href{https://robwschlegel.github.io/Intro_R_Workshop/}{Intro R
Workshop: Data Manipulation, Analysis and Graphing} for why we feel
strongly that you use R \citep{R2017} for the analyses that we will
perform here. All of the reasons provided there are valid here too, but
one reason perhaps more so than others --- R and RStudio promote the
principles of \emph{reproducible research}, and in fact make it very
easy to implement. We will focus on some of these principles throughout
the workshop, and the assignments will in fact require that you submit a
fully functional working script, complete with all the notes, memos,
examples, data, executable code, and output that will result from
completing the course material. Full, detailed reproducibility as per
the British Ecological Society
\href{https://www.britishecologicalsociety.org/wp-content/uploads/2017/12/guide-to-reproducible-code.pdf}{guidelines}
is difficult and requires advanced computing skills, but don't panic! We
will not go in to excessive amounts of detail about reproducible
research at the expense of learning about ecological sampling and data
analysis. For an example of fully reproducible research, see a recent
publication,
\href{https://github.com/ajsmit/Seaweeds_in_Two_Oceans}{Seaweeds in Two
Oceans: Beta-diversity}.

Perhaps even more fundamental for our purpose (and also in aid of
reproducibility) are the skills around data management. A large portion
of our \enquote{workflow} will concern getting the data into a format
that is suitable for analysis. And an earlier step is just as important
--- that is data entry. Sometimes we have the good fortune to enter our
data into a \enquote{tidy} format, but frequently we are given a data
set that was put together by one person or multiple people, often over
many years. These data are not so easy to coerce into a format that can
easily by analysed. We will teach you the skills needed to produce a
properly formatted data set from scratch, and provide tips for how to
transform an existing untidy data set into a tidy one. The philosophy of
tidy data is being formalised into a collection of R packages contained
within Hadley Wickham's
\href{https://www.tidyverse.org/}{\texttt{tidyverse}}, and this
collection of packages will inform the principles behind our data
transformations in order to arrive at a tidy data set, which can then be
subjected to the quantitative ecological methods (an array of
multivariate methods). Thanksfully, many of these data processing
principles have also been summarised by the British Ecological Society,
and it can be downloaded
\href{https://www.britishecologicalsociety.org/wp-content/uploads/2017/06/BES-Data-Guide-2017_web.pdf}{here}.
For the purpose of this workshop, please make sure that you have access
to both of these publications, as we will refer to them frequently,
espcially during the first few contact sessions.

What other oprions are there for analysing the kinds of data that we
will encounter in ecological research? Software packages like the ones
you may be familiar with, such as Statistica and SPSS, as well as
specilised software such as Primer (\textbf{P}lymouth \textbf{R}outines
\textbf{I}n \textbf{M}arine \textbf{E}cological \textbf{R}esearch;
equally suited to data of terrestrial origin), are often used to perform
many of the analyses we will encounter. They are rather limited with
regards to the full scope of modern multivariate statistical methods in
use by ecologists today. This is why we prefer to use R as the
\emph{engine} within which to do our ecological data analysis. R is used
by academic statisticians the world over, as well as those academic
ecological statisticians who are responsible for developing the methods
used for answering today's ecological questions. That package is called
\textbf{\texttt{vegan}} \citep{vegan2017}, and it is the one we will use
here.

\subsection*{\texorpdfstring{Some negatives of using
\texttt{vegan}}{Some negatives of using vegan}}\label{some-negatives-of-using-vegan}
\addcontentsline{toc}{subsection}{Some negatives of using
\texttt{vegan}}

Although there are many positives of using \textbf{\texttt{vegan}},
there are some negatives:

\begin{enumerate}
\def\labelenumi{\arabic{enumi}.}
\item
  It can have a steep learning curve for those whom do not like
  statistics or data manipulation, and it does require frequent use to
  remain familiar with it and to develop advanced skills
\item
  Error trapping can be confusing and frustrating
\item
  Rudimentary debugging, although there are some packages available to
  enhance the process
\item
  Handles large datasets (100 MB), but can have some trouble with
  massive datasets (GBs)
\item
  Some simple tasks can be tricky to do in R
\item
  There are multiple ways of doing the same thing
\end{enumerate}

\subsection*{The challenge: learning to program in R and
vegan}\label{the-challenge-learning-to-program-in-r-and-vegan}
\addcontentsline{toc}{subsection}{The challenge: learning to program in
R and vegan}

The same challenges as we highlighted in
\href{https://robwschlegel.github.io/Intro_R_Workshop/}{Intro R
Workshop: Data Manipulation, Analysis and Graphing} apply here, but we
still maintain that these are far outweighed by the benefits.

\subsection*{Installing R and RStudio}\label{installing-r-and-rstudio}
\addcontentsline{toc}{subsection}{Installing R and RStudio}

We assume that you already have R installed on your computer, as all of
you will have already completed the the Intro R Workshop. If you need a
refresher, please refer to
\href{https://robwschlegel.github.io/Intro_R_Workshop/}{Intro R
Workshop: Data Manipulation, Analysis and Graphing} for the installation
instructions.

--- Cheers, AJ and Robert

\section*{Resources}\label{resources}
\addcontentsline{toc}{section}{Resources}

\subsection*{Required reading}\label{required-reading}
\addcontentsline{toc}{subsection}{Required reading}

A significant amount of self study will be required to complete this
module. The content of this course follows the content of the excellent
book by Borcard, D., Gillet, F., \& Legendre, P, (2011)
\href{http://www.springer.com/gp/book/9781441979759}{Numerical ecology
with R}. There is a hardcopy in the Library.

\subsection*{Resources about multivatiate ecological
methods}\label{resources-about-multivatiate-ecological-methods}
\addcontentsline{toc}{subsection}{Resources about multivatiate
ecological methods}

To supplement the required reading, please refer to these helpful
websites:

\begin{itemize}
\tightlist
\item
  \href{http://ordination.okstate.edu/overview.htm}{Ordination Methods -
  an overview} --- as the title says, and overview of ordination methods
\item
  \href{http://ordination.okstate.edu/}{Ordination Methods for
  Ecologists} --- primarily a focus on ordination methods used by
  ecologists
\item
  \href{https://sites.google.com/site/mb3gustame/}{GUide to STatistical
  Analysis in Microbial Ecology (GUSTA ME)} --- although about
  micobiological ecology, the techniques are the same ones that will be
  used for the study of larger multicellular species
\end{itemize}

\subsection*{General resources about R}\label{general-resources-about-r}
\addcontentsline{toc}{subsection}{General resources about R}

Below you can find the source code to some books and other links to
websites about R. With some of the technical skills you'll learn in this
course you'll be able to download the source code, compile the book on
your own computer and arrive at the fully formatted (typeset) copy of
the books that you can purchase for lots of money:

\begin{itemize}
\tightlist
\item
  \href{https://github.com/hadley/ggplot2-book}{ggplot2. Elegant
  Graphics for Data Analysis} --- the R graphics bible
\item
  \href{http://shinyapps.org/apps/RGraphCompendium/index.php}{A
  Compendium of Clean Graphs in R. Version 2.0} --- using R's base
  graphics
\item
  \href{http://r4ds.had.co.nz/workflow-basics.html}{R for Data Science}
  --- data analysis using tidy principles
\item
  \href{http://rmarkdown.rstudio.com}{R Markdown} --- reproducible
  reports in R
\item
  \href{https://bookdown.org/yihui/bookdown}{bookdown: Authoring Books
  and Technical Documents with R Markdown} --- writing books in R
\item
  \href{https://shiny.rstudio.com}{Shiny} --- interactive website driven
  by R
\end{itemize}

\section*{Style and code conventions}\label{style-and-code-conventions}
\addcontentsline{toc}{section}{Style and code conventions}

Early on, develop the habit of unambiguous and consistent style and
formatting when writing your code, or anything else for that matter. Pay
attention to detail and be pedantic. This will benefit your scientific
writing in general. Although many R commands rely on precisely formatted
statements (code blocks), style can nevertheless to \emph{some extent}
have a personal flavour to it. The key is \emph{consistency}. In this
book we use certain conventions to improve readability. We use a
consistent set of conventions to refer to code, and in particular to
typed commands and package names.

\begin{itemize}
\tightlist
\item
  Package names are shown in a bold font over a grey box, \emph{e.g.}
  \textbf{\texttt{tidyr}}.
\item
  Functions are shown in normal font followed by parentheses and also
  over a grey box , \emph{e.g.} \texttt{plot()}, or \texttt{summary()}.
\item
  Other R objects, such as data, function arguments or variable names
  are again in normal font over a grey box, but without parentheses,
  \emph{e.g.} \texttt{x} and \texttt{apples}.
\item
  Sometimes we might directly specify the package that contains the
  function by using two colons, \emph{e.g.} \texttt{dplyr::filter()}.
\item
  Commands entered onto the R command line (console) and the output that
  is returned will be shown in a code block, which is a light grey
  background with code font. The commands entered start at the beginning
  of a line and the output it produces is preceded by
  \texttt{R\textgreater{}}, like so:
\end{itemize}

\begin{Shaded}
\begin{Highlighting}[]
\KeywordTok{rnorm}\NormalTok{(}\DataTypeTok{n =} \DecValTok{10}\NormalTok{, }\DataTypeTok{mean =} \DecValTok{0}\NormalTok{, }\DataTypeTok{sd =} \DecValTok{13}\NormalTok{)}
\end{Highlighting}
\end{Shaded}

\begin{verbatim}
R>  [1]  -5.3524779  11.4286604  -5.3412513
R>  [4]   8.7791273  -0.5898025 -13.2640238
R>  [7]   4.0199296  10.0253287 -38.0548744
R> [10]   7.2452439
\end{verbatim}

Consult these resources for more about R code style :

\begin{itemize}
\tightlist
\item
  \href{https://google.github.io/styleguide/Rguide.xml}{Google's R style
  guide}
\item
  \href{http://style.tidyverse.org}{The tidyverse style guide}
\item
  \href{http://adv-r.had.co.nz/Style.html}{Hadley Wickham's advanced R
  style guide}
\end{itemize}

We can also insert maths expressions, like this
\(f(k) = {n \choose k} p^{k} (1-p)^{n-k}\) or this:
\[f(k) = {n \choose k} p^{k} (1-p)^{n-k}\]

\chapter{Introduction}\label{intro}

\begin{quote}
\emph{\enquote{We have become, by the power of a glorious evolutionary
accident called intelligence, the stewards of life's continuity on
earth. We did not ask for this role, but we cannot abjure it. We may not
be suited to it, but here we are.}}

--- Stephen J. Gould
\end{quote}

\section{Modern ecological problems}\label{modern-ecological-problems}

\section{Where do ecological data come
from?}\label{where-do-ecological-data-come-from}

\subsection{Field sampling}\label{field-sampling}

\subsection{Historical data}\label{historical-data}

\subsection{Remorely sensed data}\label{remorely-sensed-data}

\subsection{Modelled data
(projections)}\label{modelled-data-projections}

\section{What do we do with these
data?}\label{what-do-we-do-with-these-data}

\subsection{Initial data entry}\label{initial-data-entry}

\subsection{Data management}\label{data-management}

\subsection{Meta-data}\label{meta-data}

\subsection{Pre-processing and quality
assurance}\label{pre-processing-and-quality-assurance}

\subsection{Analysis}\label{analysis}

\subsection{Reporting}\label{reporting}

\chapter{Ecological data}\label{ecological-data}

\begin{quote}
\emph{\enquote{The great challenge of the twenty-first century is to
raise people everywhere to a decent standard of living while preserving
as much of the rest of life as possible.}}

--- Edward O. Wilson
\end{quote}

\section{Field sampling}\label{field-sampling-1}

\section{Historical data}\label{historical-data-1}

\section{Remorely sensed data}\label{remorely-sensed-data-1}

\section{Modelled data (projections)}\label{modelled-data-projections-1}

\bibliography{LaTeX/bibliography.bib,LaTeX/packages.bib}


\end{document}
